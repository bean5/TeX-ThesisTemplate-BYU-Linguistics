\chapter{Introduction}
This is an example of the introduction. It's pretty simple and shows off
some of the basic commands.

\section{First Section}
This part shows how you can divide things into sections.

\subsection{First Subsection}
Also into subsections.

\subsubsection{First Subsubsection}
And even Subsubsections but they don't work correctly with Chapters and so I
would recommend against using them.

\subsection{Second Subsection}
Which really helps organization and automatically gets added to the Table
of Contents and gets linked to by the hyperref package.

\section{Citation Example}
One of the coolest part about \LaTeX\ is BibTeX. You can just call
the $\backslash$cite command and it will do all of the bibliography
stuff for you as long as there is an entry in the refs.bib file.
Here's an example of citing previous works
\cite{SomeSweetBook06,SomeSweetArticle06}.

\section{Math and Equation Example}
Here's how to use inline math mode to define lambda like this,
$\lambda$, and how to declare Equations~\eqref{eqn:definition_Ix}
and~\eqref{eqn:definition_Iy}

\begin{equation} \label{eqn:definition_Ix}
I_x(x,y) = \pd{I(x,y)}{x},
\end{equation}

\begin{equation}\label{eqn:definition_Iy}
I_y(x,y) = \pd{I(x,y)}{y}.
\end{equation}

If you don't want equation numbers, use
\[
\text{sign}(x) = \begin{cases}
                 1,  &\quad x> 0 \\
                 -1, &\quad x<0 \\
                 0,  &\quad \text{otherwise}
                 \end{cases}.
\]


 Or you can create equation arrays like
\begin{align}
  \alpha &= \beta^\gamma \notag \\
  x &= \frac{1}{\alpha} \label{eq:cool_1} \\
  y &= \sqrt{\abs{\frac{\gamma}{\beta}}} \notag \\
  \zeta &= x^y \label{eq:cool_2}.
\end{align}


The lines in the array can be referenced by saying things like: In
Eq.~\eqref{eq:cool_1} we show a cool equation, but its not nearly as
cool as Eq.~\eqref{eq:cool_2}.

\section{Fixed Width Figure Example} \label{sec:intro_figure_example}
This part also shows how to include a basic figure like the one
shown in Figure~\ref{fig:intro_stuff}

\begin{figure}[hhhhhtb]
  \centering
  \includegraphics[width=5.5in,natwidth=610,natheight=642]{figures/intro/stuff.jpg}
  \caption[Example Fixed Width Figure]{
This figure is just a simple figure with a width set at 5.5in. An example of a
figure whose size depends on the width of the page is given in
Figure \ref{fig:appendix_some_pic} in Section \ref{sec:appendxia_figure_example}}
%
  \label{fig:intro_stuff}
\end{figure}

\section{All Done}
You know have seen a lot of the basics and now you can see some
other fancy stuff in Chapter~\ref{chp:chapter2}.
